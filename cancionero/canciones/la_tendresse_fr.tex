\beginsong{La tendresse}[by={Marie Laforêt},
                     sr={Francés},
                     cr={},
                     index={}]

\beginverse
On peut vivre sans richesse
Presque sans le sous
Des seigneurs et des princesses
Y en a plus beaucoup
\endverse

\beginverse
Mais vivre sans tendresse
On ne le pourrait pas
Non, non, non, non
On ne le pourrait pas
\endverse

\beginverse
On peut vivre sans la gloire
Qui ne prouve rien
Être inconnu dans l'histoire
Et s'en trouver bien
\endverse

\beginverse
Mais vivre sans tendresse
Il n'en n'est pas question
Non, non, non, non
Il n'en est pas question
\endverse

\beginverse
Quelle douce faiblesse
Quel joli sentiment
Ce besoin de tendresse
Qui nous vient en naissant
Vraiment, vraiment, vraiment
\endverse

\beginverse
Le travail est nécessaire
Mais s'il faut rester
Des semaines sans rien faire
Hé bien, on s'y fait
\endverse

\beginverse
Mais vivre sans tendresse
Le temps vous paraît long
Non, non, non, non
Le temps vous paraît long
\endverse

\beginverse
Dans le feu de la jeunesse
Naissent les plaisirs
Et l'amour fait des prouesses
Pour nous éblouir
\endverse

\beginverse
Oui, mais sans la tendresse
L'amour ne serait rien
Non, non, non, non
L'amour ne serait rien
\endverse

\beginverse
Quand la vie impitoyable
Vous tombe dessus
On n'est plus qu'un pauvre diable
Broyé et déçu
\endverse

\beginverse
Alors sans la tendresse
D'un cœur qui nous soutient
Non, non, non, non
On n'irait pas plus loin
\endverse

\beginverse
Un enfant vous embrasse
Parce qu'on le rend heureux
Tout nos chagrins s'effacent
On a les larmes aux yeux
Mon dieu, mon dieu, mon dieu
\endverse

\beginverse
Dans votre immense sagesse
Immense ferveur
Faites donc pleuvoir sans cesse
Au fond de nos cœurs
\endverse

\beginverse
Des torrents de tendresse
Pour que règne l'amour
Règne l'amour
Jusqu'à la fin des jours
\endverse

\endsong