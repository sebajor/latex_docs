\beginsong{Mujeres}[
        by={Silvio Rodriguez},
        sr={},
        cr={},
        index={}]

\beginverse
Me estremeció la mujer que empinaba a sus hijos
Hacia la estrella de aquella otra madre mayor
Y como los recogía del polvo teñidos
Para enterrarlos debajo de su corazón
\endverse

\beginverse
Me estremeció la mujer del poeta, el caudillo
Siempre a la sombra y llenando un espacio vital
Me estremeció la mujer que incendiaba los trillos
De la melena invencible de aquel alemán
\endverse

\beginverse
Me estremeció la muchacha
Hija de aquel feroz continente
Que se marchó de su casa
Para otra de toda la gente
\endverse

\beginverse
Me han estremecido un montón de mujeres
Mujeres de fuego, mujeres de nieve
Y me han estremecido un montón de mujeres
Mujeres de fuego, mujeres de nieve
\endverse

\beginverse
Pero lo que me ha estremecido
Hasta perder casi el sentido
Lo que a mí más me ha estremecido
Son tus ojitos, mi hija, son tus ojitos divinos
\endverse

\beginverse
Pero lo que me ha estremecido
Hasta perder casi el sentido
Lo que a mí más me ha estremecido
Son tus ojitos, mi hija, son tus ojitos divinos
 \endverse

\beginverse
Me estremeció la mujer que parió once hijos
En el tiempo de la harina y un kilo de pan
Y los miró endurecerse mascando garifos
Me estremeció porque era mi abuela además
\endverse

\beginverse
Me estremecieron mujeres
Que la historia anotó entre laureles
Y otras desconocidas, gigantes
Que no hay libro que las aguante
\endverse

\beginverse
Me han estremecido un montón de mujeres
Mujeres de fuego, mujeres de nieve
Y me han estremecido un montón de mujeres
Mujeres de fuego, mujeres de nieve
\endverse

\beginverse
Pero lo que me ha estremecido
Hasta perder casi el sentido
Lo que a mí más me ha estremecido
Son tus ojitos, mi hija, son tus ojitos divinos
\endverse

\beginverse
Pero lo que me ha estremecido
Hasta perder casi el sentido
Lo que a mí más me ha estremecido
Son tus ojitos, mi hija, son tus ojitos divinos
\endverse

\endsong