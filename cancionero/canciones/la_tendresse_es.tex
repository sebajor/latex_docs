\beginsong{La tendresse}[by={Marie Laforêt},
                     sr={Español},
                     cr={},
                     index={}]

\beginverse
Puedo vivir sin riquezas,
Y pobre también,
De señores y princesas,
Muy poco se ven.
\endverse


\beginverse
Más vivir sin ternura,
Yo no podré jamás,
No, no, no, no,
Yo no podré jamás.
\endverse

\beginverse
Puedo vivir sin la gloria,
Que no es primordial,
Sin que hablen en la historia,
Lo mismo me da.
\endverse

\beginverse
Más vivir sin ternura,
Que amarga sensación,
No, no, no, no,
Que amarga sensación.
\endverse

\beginverse
Que dulce es el querer,
Qué bonito es sentir,
Desear la ternura,
Que nos viene al nacer.
\endverse

\beginverse
 Verdad, verdad, verdad.
\endverse

\beginverse
La juventud en su fuego,
Puede conocer,
Que al amor rinde su juego,
Con su gran placer.
\endverse

\beginverse
Pero sin la ternura,
Es imposible amar,
No, no, no, no,
Es imposible amar.
\endverse

\beginverse
Un niño nos abraza,
Lo hacemos muy feliz,
Se marcha la tristeza,
Al verla así vivir.
\endverse

\beginverse
 Mi Dios, mi Dios, mi Dios.
\endverse

\beginverse
En nuestra inmensa armonía,
Inmenso fervor,
Debemos pedir al cielo,
Que no hay corazón.
\endverse

\beginverse
Se imponga la ternura,
Y que reine el amor,
Reine el amor,
Hasta que quiera Dios. 
\endverse


\endsong